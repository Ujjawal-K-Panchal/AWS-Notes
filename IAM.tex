\chapter{IAM 101: Identity Access Management}
allows to manage users and their level of access in AWS concole.

\section{Key concepts}
\begin{itemize}
	\item Centralized control of AWS account.
	\item shared access to your AWS account.
	\item Granular Permissions.
	\item Identity Federation.
	\item Multifactor authentication.
	\item temporary access.
	\item allows you to set up password rotation policy.
	\item integrates with different AWS services.
	\item supports PCI DSS. (Compliant framework for taking credit card details).
\end{itemize}

\section{Key terminology.}
\begin{itemize}
\item Users
\item Groups (Users belonging to group ingerit properties.)
\item Policies (Permmissions in JSON files).
\item Roles (Role that AWS service plays, eg. allowing a VM to modify data in S3).
\end{itemize}

\section{A Billing Alarm - LAB}
How to get automatic notifications when bill goes over something.
To get bill notifications, goto cloud watch, billing alarms and create SNS topic.

\section{Summary}
\begin{itemize}
	\item IAM is universal.
	\item root account is simply created when you first setup your AWS account. It has complete admin access.
	\item New Users have no permissions when first created.
	\item New users are assigned access key ID and Secret Access Keys when first created.
	\item You only get to view them once. So you will need to regenerate them.
	\item \textbf{Always setup multifactor authentication on root account.}
	\item ability to create and customize password rotation policy.	
\end{itemize}